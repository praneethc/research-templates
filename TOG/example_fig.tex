%\begin{figure}[htb]
\begin{teaserfigure}
  \centering
  \subfloat[raster]{
    \label{fig:example:raster}
    \includegraphics[width=0.2\linewidth]{example-image-a}
  }%subfloat
  \subfloat[vector]{
    \label{fig:example:svg}
    \includegraphics[width=0.29\linewidth]{example-image-a}
  }%subfloat
  \subfloat[vector + raster]{
    \label{fig:example:svg_img}
    \includegraphics[width=0.45\linewidth]{example-image-a}
  }%subfloat
 \Caption{Example figure.}
 {%
\subref{fig:example:raster} is a raster image and \subref{fig:example:svg} is a vector graphics.
Never, ever, rasterize vector graphics unless you want large size and low quality files.
We can also combine vector and raster graphics as in \subref{fig:example:svg_img}.
Use Inkscape to import  an image into an svg file, and draw over it using whatever stuff like texts or strokes.
Make sure the image is linked not embedded to avoid duplicating the image, and the link should be relative in the same path \cite{StackExchange:2011:HLI}.
 }
 \label{fig:example}
%\end{figure}
\end{teaserfigure}
